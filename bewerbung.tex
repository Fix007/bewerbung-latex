% Copyright 2009-2011,2015-2019 Dominik Wagenfuehr <dominik.wagenfuehr@deesaster.org>
% Dieses Dokument unterliegt der Creative-Commons-Lizenz
% "Namensnennung-Weitergabe unter gleichen Bedingungen 4.0 International"
% [http://creativecommons.org/licenses/by-sa/4.0/deed.de].
%
% Beispiel-Bewerberfoto: Copyright 2013 TVJunkie
% https://commons.wikimedia.org/wiki/File:Gnome-Wikipedia-user-female.png
% Lizenz: Creative-Commons-Lizenz "Attribution-Share Alike 3.0 Unported"
% [https://creativecommons.org/licenses/by-sa/3.0/deed.en]

\documentclass[fontsize=12pt,parskip=half-]{scrartcl}

% Liest die vordefinierten Befehle ein. Nicht veraendern!
\usepackage{bewerbung-latex}

% Optionales Paket, falls jemand \Telefon und \Email
% als Symbol nutzen will.
%\usepackage{marvosym}

%%%%%%%%%%%%%%%%%%%%%%%%%%%%%%%%%%%%%%%%%%%%%%%%%%%
% Ab hier sollte man selbst Aenderungen vornehmen.
%%%%%%%%%%%%%%%%%%%%%%%%%%%%%%%%%%%%%%%%%%%%%%%%%%%

% Andere Schrift festlegen (nur machen, falls man die korrekten Namen kennt).
% Das erste Argument ist der Paketname für pdfTeX, das zweite Argument
% der Schriftname für XeTeX und LuaTeX.
% Als Standard wir hier TeX Gyre Pagella benutzt (Paket "tex-gyre").
\SetzeSchrift{tgpagella}{TeX Gyre Pagella}     % optional

% Umschalten auf serifenlose Schrift. Wenn gewuenscht, einkommentieren.
% Achtung: Nicht jede serifenlose Schrift kommt mit den Kapitaelchen in
% den Ueberschriften (Lebenslauf, Zu meiner Person etc.) klar. Sollten
% diese mit Serifen angezeigt werden, dann muss man zusaetzlich das
% optionale Argument [noscshape] angeben. Dann gibt es natuerlich auch
% keine Kapitaelchen!
%\NutzeSerifenlosenSchrift[]{}%

% Zuerst die Sprache festlegen.
% Per Standard ist "Deutsch" gesetzt.
% Zusaetzliche, moegliche Angaben sind: "Schweiz"
%\SetzeSprache{Schweiz}          % optional

% Eigene Daten festgelegt.
\VollerName{Eva Mustermann}
\AbsenderStrasse{Musterweg 42}
\AbsenderPLZOrt{12345 Berlin}
% Als erstes Argument kann man beim Telefon eine Angabe
% davor angeben. Der Wert darf auch leer sein.
% Bei Telefonnummer ggf. Laenderkennung ergaenzen, wenn
% die Telefonnummer nicht im Land des Empfaengers ist.
\bwTelefon{Tel.:}{0170 567890}
%\bwTelefon{\Telefon}{+49 170 567890}
% Als erstes Argument kann man bei der E-Mail eine Angabe
% davor angeben (z.B. "E-Mail:". Der Wert darf auch leer sein.
% Zusaetzlich kann optional eine Farbe angegeben werden,
% mit der die Mailadresse dargestellt werden soll.
% Standard ist "black"  als Farbe. Beispiel: \EMail[orange]{...}{...}
% Achtung: Die Farbe wird fuer alle Links benutzt, damit es einheitlich aussieht.
\bwEMail[Blue]{}{eva.mustermann@musterstadt.de}
%\bwEMail{\Email}{eva.mustermann@musterstadt.de}
\OrtDatum{Berlin, \HeutigerTag}
\Geburtstag{1. Januar 1985}
\Geburtsort{Berlin}
% Der Ausbildungsgrad und die Details sind optional.
% Leer lassen oder auskommentieren, wenn nichts angezeigt werden soll.
\Details{ledig, ortsungebunden} % verheiratet, 2 Kinder, etc.
\Ausbildungsgrad{Promovierte Veterinärmedizinerin}    % optional
% Optionales Xing- und Linkedin-Profil
% Beides wird nur auf der Persoenlichen Seite angezeigt,
% nicht im Anschreiben.
\XingProfil{Eva\_Mustermann}
\LinkedInProfil{evamustermann}

% Die Unterschrift sollte als Bilddatei vorliegen.
% Leer lassen oder auskommentieren, wenn keine Signatur eingebunden
% werden soll.
% Als Option kann man die Breite der Unterschrift angeben.
% Eine Breite von 5cm ist der Standard.
\UnterschriftenDatei[5cm]{signatur-bewerber.png}           % optional

% Das Bewerberfoto.
% Leer lassen oder auskommentieren, wenn keine Foto auf
% der "Meine Seite"-Seite angezeigt werden soll.
% Als Option gibt man die Hoehe des Bildes an. Eine Hoehe von 10.5 cm
% ist das groesste, was noch auf die Seite passt.
\BewerberFoto[6cm]{foto-bewerber.jpg}                 % optional

% Adressat festlegen
% Die Abteilung und der Vorname sind optional und koennen leer
% gelassen oder auskommentiert werden.
% Wenn der Nachname leergelassen oder auskommentiert wird,
% wird die Anrede automatisch durch "Damen und Herren" ersetzt.
\Firma{Hell AG}
\Abteilung{Personalabteilung}            % optional
\AdressatVorname{Christina}              % optional
\AdressatNachname{Funkel}                % ggf. optional
\AnschriftStrasse{Route 66}
\AnschriftPLZOrt{00000 Havenfürst}
% Als optionales Argument hat der Titel eine Kurzform, die nur im Briefkopf
% benutzt wird. Wenn das optionale Argument fehlt, wird die normale Angabe 
% fuer Briefkopf und Anrede benutzt.
\AdressatTitel[Prof. Dr.]{Professorin}   % optional

% Achtung: Anrede muss am Ende dieser Liste stehen,
% damit dies korrekt um Titel und Name ergaenzt wird!
\Anrede{Frau}                          % alternativ: Frau, Herr oder leer lassen

% Man kann hier einen zweiten Ansprechpartner beim Anschreiben angeben.
% Die Adresszeile wird entsprechend ergaenzt.
%\ZweiterAdressatVorname{Pitti}            % optional
%\ZweiterAdressatNachname{Platsch}         % ggf. optional
%\ZweiterAdressatTitel[Dr.]{Doktor}        % optional
%\ZweiterAdressatAnrede{Herr}

% Die Stelle, auf die man sich bewirbt
\Bewerberstelle{Veterinär-Mediziner (m/w)}

% Beginn des Dokuments, nicht aendern!
\begin{document}

%%%%%%%%%%%%%%%%%%%%%%%%%%%
% Das Anschreiben
%%%%%%%%%%%%%%%%%%%%%%%%%%%

% Anschreiben und Motivationsseite sind per Standard als linksbuendiger
% Flattersatz gesetzt. Mit der nachfolgenden Option kann man auf
% Blocksatz umstellen.
%\NutzeBlocksatz{}

% Ausrichtung des Absenders im Anschreiben (nur im Anschreiben, nicht
% auf der eigenen Seite!)
% Aufgrund der Ausrichtung des Adressaten will man ggf. auch den Absender
% neu ausrichten, damit es nicht direkt übereinander steht. Per Standard:
% Deutsch: rechts
% Schweiz: links
% Aber man kann es manuell umschalten, wenn man will.
%\AbsenderAusrichtung{rechts}          % optional

% Ausrichtung des Adressaten im Adressfeld
% Es gibt Laender, wo das Adressfeld nicht links,
% sondern rechts steht. Per Standard gilt:
% Deutsch: links
% Schweiz: rechts
% Aber man kann es manuell umschalten, wenn man will.
%\AddressatAusrichtung{links}         % optional

% Abstand zwischen dem Absender und dem Adressat im Anschreiben.
% Gemessen wird in Zeilen, d.h. der Wert 1.5 steht fuer anderthalb Zeilen.
% Der Standard-Wert ist 0.
%\AbstandZwischenAdressen{0}      % optional

% Abstand vor dem eigentlichen Anschreiben (inkl. Ort und Datum).
% Gemessen wird in Zeilen, d.h. der Wert 1.5 steht fuer anderthalb Zeilen.
% Der Standard-Wert ist 1.
\AbstandVorAnschreiben{3}

% Anschreibenseite vergroessern, d.h. es ist damit moeglich, ueber den
% eigentlichen unteren Rand zu schreiben, falls das Anschreiben etwas
% laenger geworden ist.
% Gemessen wird in Zeilen, d.h. der Wert 1.5 steht fuer anderthalb Zeilen.
% Der Standard-Wert ist 0. Als Maximalwert sollte man 3 einstellen, ansonsten
% wirkt das Anschreiben vom Aufbau sehr unausgeglichen.
\AnschreibenSeiteVergroessern{0}

% Abstand zwischen Unterschrift und den Anlagen im Anschreiben
% Gemessen wird in Zeilen, d.h. der Wert 1.5 steht fuer anderthalb Zeilen.
% Der Standard-Wert ist 1.
\AbstandVorAnlagen{2}

% Hinweis auf Anlagen
% Wenn nicht benoetigt, dann einfach auskommentieren.
%\AnschreibenAnlage{Anlagen}

\begin{Anschreiben}
    % Hinweis Anfang – nach dem Lesen loeschen!
    % In der Schweiz beginnt man den Satz im Uebrigen gross, da kein
    % Komma bei der Anrede benutzt wird.
    hier steht mein Bewerbungstext, wieso ich mich auf die Stelle bewerbe
    und bei der Firma anfangen möchte. Der Text sollte nicht zu lang sein
    und nur die wichtigsten Details enthalten.
    
    Den Hinweis hier muss ich natürlich löschen! Alles, was in der Vorlage
    zwischen den Kommentarzeilen steht, kann weg.
    % Hinweis Ende – nach dem Lesen loeschen!
\end{Anschreiben}

% Optional kann man die Farbe für die Ueberschriften ab hier festlegen.
% Dazu zaehlt auch die eigene Seite.
% Als Optionen kann man die normalen LaTeX-Farben verwenden (Englisch):
% https://en.wikibooks.org/wiki/LaTeX/Colors
% Per Standard wird "Black" (Schwarz) verwendet.
% Achtung: Man sollte es nicht zu bunt treiben!
%\UeberschriftFarbe{Black}            % optional

%%%%%%%%%%%%%%%%%%%%%%%%%%%%%%%%%%%%%%%%%%%
% Meine Seite
%%%%%%%%%%%%%%%%%%%%%%%%%%%%%%%%%%%%%%%%%%%

\MeineSeite

%%%%%%%%%%%%%%%%%%%%%%%%%%%%%%%%%%%%%%%%%%%
% Lebenslauf
%%%%%%%%%%%%%%%%%%%%%%%%%%%%%%%%%%%%%%%%%%%

% Ausrichtung der Ueberschriften.
% Man kann die Ausrichtung prinzipiell vor jedem Kapitel
% (Lebenslauf, Motivation, Anlagen) neu einstellen, aber gleichförmiger
% ist es, wenn man es nur einmal hier definiert.
% Moegliche Angaben sind: links, rechts, mittig.
% Per Standard wird "rechts" ausgewaehlt.
\UeberschriftAusrichtung{rechts}         % optional

% Schriftgroesse der Ueberschriften
% Erlaubt sind alle Standard-Schriftgroessenangaben.
% Sinnvoll sind \LARGE, \huge, \Huge
% Standard ist \LARGE
\UeberschriftGroesse{\LARGE}            % optional

% Ueberschrift des Lebenslaufes.
% Hier kann man auch zum Beispiel "Curriculum Vitae" eintragen.
% Per Standard wird "Lebenslauf" ausgegeben.
\UeberschriftLebenslauf{Lebenslauf}     % optional

% Der Lebenslauf ist in Abschnitte unterteilt. Jeder Abschnitt hat dabei
% einen Titel. Innerhalb des Abschnitts (zwischen "begin" und "end")
% legt man mit \EintragCV einen Eintrag an.
% In der Textgestaltung ist man dabei ansonsten frei.
% Legt den Einschub des Inhalts eines CV-Abschnitte (d.h. der Tabelle) fest.
% Standard ist 8pt.
% Das optionale Argument "buendig" gibt an, dass die Ueberschrift zusaetzlich
% noch buendig zur Tabelle sein soll und nicht linksbuendig per Standard.
\EinschubCV[buendig]{8pt}            % optional
%\EinschubCV{8pt}

% Optional kann man die Farbe fuer die einzelnen Abschnitte
% und/oder fuer die Linien darunter festlegen.
% Als Optionen kann man die normalen LaTeX-Farben verwenden (Englisch):
% https://en.wikibooks.org/wiki/LaTeX/Colors
% Per Standard wird "Black" (Schwarz) verwendet.
% Achtung: Man sollte es nicht zu bunt treiben!
\AbschnittFarbe{Blue}                  % optional
\AbschnittLinienFarbe{Blue}            % optional

\begin{Lebenslauf}

\begin{AbschnittCV}{Promotion}%
% Das erste Argument ist die Dauer, das zweite die Bezeichnung.
% Wichtig: Wenn man mehrere Zeilen zu einem Eintrag hat, den Umbruch mit
% \newline erzwingen und nicht mit \\, weil intern eine Tabelle benutzt wird!
\EintragCV{Mai 2015}{Doktor der Veterinärmedizin Dr.\,vet. \newline
                     Doktorarbeit an der Freien Universität Berlin: \newline
                     „\textit{Habitatsverhalten von Meerschweinchen unter Einfluss von Halluzinogenen}“
}
\end{AbschnittCV}

% Ein optionales Argument von "AbschnittCV" gibt an wie viel Platz
% unterhalb des gesamten Abschnitts gelassen wird.
% Gemessen wird in Zeilen, d.h. der Wert 1.5 steht fuer anderthalb Zeilen.
% Der Standard-Wert ist 0.5, also eine halbe Zeile.
\begin{AbschnittCV}[0]{Studium}%
% Ein optionales Argument von "EintragCV" gibt an wie viel Platz
% unterhalb des Eintrags gelassen wird.
% Gemessen wird in Zeilen, d.h. der Wert 1.5 steht fuer anderthalb Zeilen.
% Der Standard-Wert ist 0.5, also eine halbe Zeile.
\EintragCV[1]{September 2012}{Abschluss als Diplom-Veterinärmedizinerin (Bewertung „sehr gut“) \newline
                           Diplomarbeit an der Freien Universität Berlin: \newline
                           „\textit{Motorische Analyse der Klapperschlange – Bewegung und Fortpflanzung}“
}
\EintragCV{2006 – 2012}{Studium Diplom-Veterinärmedizin \newline
                        an der Freien Universität Berlin \newline
                        Schwerpunkte: Säugende Reptilien
}
\end{AbschnittCV}

\begin{AbschnittCV}{Freiwilliges Soziales Jahr}%
\EintragCV{2005 – 2006}{Malteser-Krankenhaus Berlin}
\end{AbschnittCV}

\begin{AbschnittCV}{Schulbildung}%
\EintragCV{Juni 2005}  {Abiturprüfung bestanden mit Note 1,3}
\EintragCV{1998 – 2005}{Gymnasium Pestalozzischule Berlin}
\end{AbschnittCV}

\begin{AbschnittCV}{Veröffentlichungen}%
\EintragCV{2010}{„\textit{Cutting Glasses with the Teeth of Crotalus adamanteus}“ für „Ninth International Conference of World Wide \mbox{Veterinary} Medicine“ in Barcelona (Spanien), Januar 2011}
\end{AbschnittCV}

% Wenn man will, kann man hier (oder irgendwo anders) den Lebenslauf
% explizit unterbrechen und eine neue Seite beginnen. Zum Beispiel,
% wenn die Tabelle nicht gut unterbrochen wird.
\NeueSeiteAbschnittCV

\begin{AbschnittCV}{Praktische Tätigkeiten}%
\EintragCV{2003 – 2004 und \newline%
           2006 – 2012}{Tierarzt-Praxis Dr. Hauser, Berlin}
% Unterabschnitte koennen eine feinere Gliederung bestimmter Taetigkeiten bieten
\UnterabschnittCV{Kernarbeitsgebiete}{%
    \punkt Hunden die Pfötchen halten
    \punkt Katzen kraulen
}
\UnterabschnittCV{Weitere Aufgaben}{%
    \punkt Wasser besorgen
}
\EintragCV{2005}{Volontariat Udawalawe-Nationalpark, Sri Lanka}
\UnterabschnittCV{Kernarbeitsgebiete}{%
    \punkt Elefanten füttern
}
\end{AbschnittCV}

\begin{AbschnittCV}{Besondere Kenntnisse}%
\EintragCV{Sprachen}{Englisch: sehr gut in Schrift und gut in Sprache \newline
                     Französisch: Grundkenntnisse \newline
                     Latein: Abschluss mit Latinum
}
\EintragCV{Medizin}{Allgemeine Veterinärmedizin \newline
                    Reptilien
}
\end{AbschnittCV}

\begin{AbschnittCV}{Hobbys und Interessen}%
\EintragCV{}{Literatur \newline
             Badminton
}
\end{AbschnittCV}

\end{Lebenslauf}

%%%%%%%%%%%%%%%%%%%%%%%%%%%%%%%%%%%%%%%%%%%
% Motivation
%%%%%%%%%%%%%%%%%%%%%%%%%%%%%%%%%%%%%%%%%%%

% Text fuer die Motivationsseite
\begin{Motivation}
    % Hinweis Anfang – nach dem Lesen loeschen!
    Hier steht mein Motivationstext. Ich kann ein bisschen mehr
    über mich erzählen und wieso ich bei der Firma anfangen will,
    wenn ich das noch nicht im Anschreiben getan habe.
    
    Dabei kann ich auch etwas weiter ausholen, frühere Tätigkeiten
    beschreiben, aber auch auf private nützliche Dinge eingehen,
    die mir bei der Erfüllung der Aufgabe helfen können.
    
    In dem Beispiel würde ich speziell auf die Erfahrungen in der
    Tierartzpraxis eingehen und die eigenen Haustiere erwähnen.
    
    Den Hinweis hier muss ich natürlich löschen! Alles, was in der Vorlage
    zwischen den Kommentarzeilen steht, kann weg.
    % Hinweis Ende – nach dem Lesen loeschen!
\end{Motivation}

%%%%%%%%%%%%%%%%%%%%%%%%%%%%%%%%%%%%%%%%%%%
% Anlagen
% Wenn man keine Anlagen hat, dann alles
% auskommentieren oder loeschen.
%%%%%%%%%%%%%%%%%%%%%%%%%%%%%%%%%%%%%%%%%%%

% Bei den Anlagen kann man als Option einen Dateinamen angeben,
% der identisch zu \AnlageEinfuegen unten ist. Dann wird die Seite
% im Verzeichnis entsprechend verlinkt und der Eintrag unterstrichen.

\begin{Anlagenverzeichnis}
    \AbschnittAnlage{Arbeitszeugnisse}
    \begin{Auflistung}
        \Anlage{Tierarzt-Praxis Dr. Hauser}
        \Anlage{FSJ, Malteser-Krankenhaus Berlin}
    \end{Auflistung}

    \AbschnittAnlage{Zeugnisse}
    \begin{Auflistung}
        \Anlage{Doktorurkunde}
        \Anlage[diplomurkunde.pdf]{Diplomurkunde}
        \Anlage{Diplomzeugnis}
        \Anlage{Abiturzeugnis}
    \end{Auflistung}
\end{Anlagenverzeichnis}

% Man kann die Anlagen auch direkt in das Dokument einbinden,
% wenn sie als PDF vorliegen. Oder man versendet sie separat.
% Mit der Option "quer" wird die Anlage im Querformat eingebunden.
%\AnlageEinfuegen{diplomurkunde.pdf}
%\AnlageEinfuegen[quer]{doktorurkunde.pdf}

% Ende des Dokuments, nicht aendern!
\end{document}
